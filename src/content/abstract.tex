\pagenumbering{roman}
\setcounter{page}{1}

\selecthungarian

%----------------------------------------------------------------------------
% Abstract in Hungarian
%----------------------------------------------------------------------------
\chapter*{Kivonat}\addcontentsline{toc}{chapter}{Kivonat}

Informatikai rendszerek tervezése, fejlesztése modell-alapú technológiák segítségével hatékonyabbá tehető. A tervezett rendszerről rendelkezésre álló formális modell lehetővé teszi olyan feladatok automatizált végrehajtását, mint a helyességellenőrzés, kódgenerálás, valamint a rendszer kvalitatív és kvantitatív analízise.
\\\\
A rendszermodell reaktív komponensei (pl. kommunikációs protokollok) tipikusan állapot-alapú formalizmusokkal modellezhetők. Sokszor azonban a szükséges rendszermodell elkészítése nehéz feladat, hiszen ezen protokollokat kényelmesebb példalefutások alapján megtervezni. Tudomásom szerint az irodalomban még nincs olyan eszköz, mely lehetővé tenné egy rendszer(komponens) megtervezését példalefutások alapján.
\\\\
Munkám célja a rendszertervezés megkönnyítése, egy olyan szoftver létrehozásával, mely lehetővé teszi egy rendszer példalefutások alapján való megtervezését. A tervezett rendszer magját automatatanuló algoritmusok képzik, melyek erőssége éppen az, hogy állapot-alapú modelleket hoznak létre tervezett viselkedések alapján.
\\\\
A fenti cél megvalósítására a dolgozatomban bemutatok egy moduláris, kiterjeszthető és más szoftverekkel integrálható keretrendszert. Munkám során két algoritmust is implementáltam, melyek tetszőleges formalizmusokat képesek kezelni. Az elkészült keretrendszer hatékonyságát és alkalmazhatóságát mind elméleti szempontból, mind mérésekkel is igazoltam. 
\\
A bemutatott keretrendszer a rendszertervezés elősegítésén kívül az automatatanuló algoritmusok fejlesztését, összehasonlítását és tetszőleges célú felhasználását is lehetővé teszi.

%TODO: "olyan nincs, ami kiterjeszthető stb stb"
\vfill
\selectenglish


%----------------------------------------------------------------------------
% Abstract in English
%----------------------------------------------------------------------------
\chapter*{Abstract}\addcontentsline{toc}{chapter}{Abstract}

The design and development of technological systems can be made more efficient using model-based technologies. A formal model of the system under design makes it possible to automate tasks such as verification, code generation and qualitative, quantitative system-analysis.
\\\\
Reactive components of a system model (e.g. communication protocols) are typically modeled using state-based formalisms. However, the construction of such a system-model is often difficult, since these protocols are more straightforward to design using example runs. To the best of my knowledge, there is no tool in the literature capable of creating  a state based model of a system (component) from a set of runs and then performing the desired analysis on the output.
\\\\
The objective of my work is to ease the process of system design by creating a software, which makes it possible to model a system based on example runs. The core of this framework is provided by automaton learning algorithms, whose strength lies in the creation of state-based models using behavioral information.
\\\\
To attain the above goal, I present a modular, extensible and easily integrable framework in this thesis. During my work, I've implemented two algorithms capable of handling arbitrary formalisms. I have evaluated the efficiency and applicability of these algorithms both theoretically, and by measurements.
\\
The presented framework, while capable of supporting system design, also enables the development, comparison and application of automaton learning algorithms.



\vfill
\selectthesislanguage

\newcounter{romanPage}
\setcounter{romanPage}{\value{page}}
\stepcounter{romanPage}