\chapter{Conclusions}
This chapter concludes the contributions of this thesis, and presents my future goals.

\section{Contributions}
%saját kontribúció pontokban szedve
%tudományos kontribúció pontokban sedve
%2mondat. As a result, lehetővé tettem, van egy keretrendszer...
In this thesis, I have achieved the following theoretical and experimental results:
\begin{itemize}
	\item I designed a modular, easily extensible framework for the development of automaton learning algorithms capable of handling arbitrary formalisms.
	\item Implemented a prototype of the above framework.
	\item Created multiple in- and output formalisms, including an EMF ecore meta-model utilizing an Xtext grammar.
	\item Implemented the Direct Hypothesis Construction (DHC) active automaton learning algorithm into the framework.
	\item Implemented the TTT algorithm by depending on abstractions defined by the LearnLib\cite{10.1007/978-3-319-21690-4_32} framework. Implemented a large part of the core algorithm, while delegating to LearnLib for optimization purposes.
	\item Demonstrated the correctness of both the DHC and TTT algorithms using case studies.
	\item I have theoretically evaluated the framework and its implemented algorithms with respect to architecture and efficiency.
	\item I have experimentally evaluated the framework and compared the algorithms implemented within, examining if and how they are capable of supporting system design. I have optimized the inefficiencies uncovered by the experimental evaluation.
\end{itemize}

As a result, an extensible and modular framework for active automaton learning was created. This framework contains an implementation of the Direct Hypothesis Construction and the TTT algorithms both of which are straightforward to extend using arbitrary formalisms. Utilizing these algorithms and the formalisms already implemented in the framework, system design can be supported by modeling system (components) based on example runs, as shown in the case studies presented. The framework can also be used for comparison and practical use of learning algorithms.

\section{Future work}
%további algoritmusok, gamma intergráció közelebbről távolabbra. (keretrendszertől absztrakthozt)
In terms of future work, more existing algorithms (such as $L^*$) are to be implemented. Onboarding new input and output formalisms would also allow a wider variety of use.  An integration with the Gamma Statechart Composition framework\cite{DBLP:conf/icse/MolnarGVMV18} and the Theta framework\cite{theta-fmcad2017} is planned to enable code-generation and system analysis based on the learned automata. There are also plans for experimentation with dynamic teacher components in active automaton learning, capable of on-the-fly decision making during the learning process.